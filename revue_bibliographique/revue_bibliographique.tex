\documentclass[a4paper,11pt]{article}

\usepackage[utf8]{inputenc}
\usepackage{mathpazo}
\usepackage{gensymb}
\usepackage{graphicx}
\usepackage{url}
\usepackage[frenchb]{babel}
\usepackage{subfig}
\usepackage{apacite}
\usepackage{array}
\usepackage{multirow}
\usepackage[T1]{fontenc}
\usepackage{amsmath}
%\usepackage{natbib}
%\bibliographystyle{plainnat}
%\newcommand*{\dl}{\cline{2-2}\cline{4-4}}
%\usepackage [nottoc]{tocbibind}

\begin{document}

\title{Sorgho \& Niébé au Burkina Faso}
\author{Marilyne Aza-Gnandji}
\date{Avril 2018} 

\maketitle
\tableofcontents

\section{Productions agricoles au Burkina Faso}

Le Burkina Faso est un pays à économie
agraire\,\cite{Koulibi_FideleZONGO} et selon les statistiques de 2016
la population serait de 17\,589\,198 habitants pour une superficie de
274\,200\,km\textsuperscript{2}\,\footnote{\url{https://fr.actualitix.com/pays/bfa/statistiques-agriculture-burkina-faso.php}}. Une
publication faite le 24 avril 2017 montre que le secteur
agro-sylvo-pastoral représente 35\% du PIB burkinabé, et emploie 82\%
de la population active et que sur les 11,8 millions d'hectares de
terres à potentialités agricoles, seuls 5,7 millions d’hectares sont
cultivés\footnote{\url{http://www.economiesafricaines.com/les-territoires/burkina-faso/les-secteurs-d-activite/le-secteur-agricole}}. Aussi
l'arboriculture et le maraîchage occupent une place non
négligeable\footnote{\url{http://agriculture.gouv.fr/burkina-faso}}
avec environ 600\,000 petites exploitations agricoles qui sont par
définition dans le recensement agricole, de petites unités de
production remplissant les trois critères suivants: produire des
produits agricoles; avoir une gestion courante indépendante; atteindre
un certain seuil en superficie (inférieur ou au moins égal à un
hectare), en production ou en nombre
d’animaux\footnote{https://www.insee.fr/fr/metadonnees/definition/c1186}. La
production agricole s’articule essentiellement autour des cultures
pluviales, notamment céréalières. Les principales cultures céréalières
sont le mil, le sorgho, le maïs et le riz. A côté de ces cultures, on
retrouve les cultures vivrières (le niébé) et les cultures de rente
(le coton, l’arachide et le sésame). Les populations s’adonnent
également aux cultures de contre saison, surtout celles
maraîchères\footnote{\url{https://www.mdh-limoges.org/spip.php?article2273}}.
En 2016, la production céréalière était de 4\,567\,066 tonnes soit
près de 68\% de la production agricole et ce pour des superficies
emblavées en céréales qui s'évaluaient à 4\,017\,586\,ha. Ces
superficies emblavées en céréales sont en hausse par rapport à la
campagne de
2015\footnote{\url{www.insd.bf/n/contenu/pub_periodiques/tableaux_de_bord/TBE/TBE_BF_2017T1.pdf}}. L’activité
agricole au Burkina Faso est menée dans des conditions parfois
défavorables (mauvaise pluviométrie, pauvreté des sols, \ldots{}). Ce
faisant, la production agricole est faible et ne parvient pas à
couvrir les besoins alimentaires des populations. Le Burkina Faso
possède un climat tropical de type soudano-sahélien (caractérisé par
des variations pluviométriques considérables allant d’une moyenne de
350 mm au Nord à plus de 1\,000 mm au Sud-Ouest) avec deux saisons
très contrastées : la saison des pluies avec des précipitations
comprises entre 300\,mm et 1\,200 mm et la saison sèche durant laquelle
souffle l’harmattan, un vent chaud, sec et chargé de poussière,
originaire du Sahara. La saison des pluies dure environ 4 mois, entre
mai-juin et septembre, sa durée est plus courte au nord du
pays\footnote{http://www.burkina-faso.ca/climat-du-burkina-faso/}. Le
taux de croissance de la production agricole est de l’ordre de 4,3\%
(1983 à 2007) et, selon
\,\citeauthor{Ngaido_2006},\,\citeyear{Ngaido_2006} ce taux devrait
être de 6,8\% pour l’atteinte des Objectifs du Millénaire pour le
Développement en matière de réduction de la faim. Alors, près de 46\%
de la population totale est exposé à l’insécurité alimentaire et on
estimait en 2003 le niveau de couverture des besoins nutritionnels à 2
283 kilocalories par personne et par jour contre les 2\,500 kcal
requis. Parallèlement, la pauvreté au Burkina est essentiellement
rurale (la contribution du milieu rural à la pauvreté s’élevait à
92,2\% en 2003) et la population rurale est en grande majorité
agricole\,\cite{DPSAA_2011}. Le Burkina Faso a besoin d’une croissance
soutenue de sa production agricole non seulement pour assurer sa
sécurité alimentaire mais aussi pour lutter contre la pauvreté et
assurer son développement
économique\footnote{https://www6.inra.fr/fabatropimed/FTM-Publications/FTM-Master-BTS}.

\section{Le Sorgho: \emph{Sorghum bicolor} (L.) Moench}

% c\'{e}r\'{e}ales
%%Insérer dans un texte une référence de figure : (Fig.~\ref{images:photo_plante_sorgho})


%\begin{figure}[htbp]
 % \begin{center}
  %  \leavevmode
   % \subfloat[Plante du Sorgho]{%
    %  \label{fig-Plante du Sorgho}
    %  \includegraphics[width=8cm,angle=270]{images/Plante_Sorgho}}
   % \hspace{0.5cm}
   % \subfloat[Graines du Sorgho]{%
    %  \label{fig-Graines du Sorgho}
   %   \includegraphics[width=8cm,angle=270]{images/GrainesDeSorgho}}
  %  \caption{Plante,graines du sorgho-Schéma plant de sorgho et sa représentation en champ réel}
 % \end{center}
%\end{figure}


  
\subsection{Origine \& Diffusion}

Le sorgho est originaire de l'ancien monde et problablement de la
partie nord-est de l'Afrique où la variabilité des sorghos cultivés et
sauvages est maximale. \emph{ Sorghum bicolor (Linn.) Moench} passe
généralement pour être l'appelation spécifique la plus juste des
sorghos cultivés\,\cite{food1977introduction}. Le sorgho commun
\emph{Sorghum bicolor} est une graminée très répandue à l’état sauvage
sous les climats tropicaux et subtropicaux. C’est une plante herbacée
de climat chaud qui est cultivée pour ses graines et son fourrage. Il
ressemble un peu au maïs mais son système racinaire le rend résistant
à la sécheresse ce qui fait qu'il est principalement cultivé en
Afrique pour nourrir le bétail (sorgho fourrager) et pour
l'alimentation humaine (sorgho grain). Toutefois, on le trouve aussi
beaucoup en Inde et en Amérique du Sud, ce qui le classe au 5\ieme{} rang
des céréales, après le maïs, le riz, le blé et l'orge. Selon les
continents, il porte différents noms: \og{}gros mil\fg{} en Afrique,
\og{}millet indien\fg{} en Asie,\og{}blé égyptien\fg{} au
Moyen-Orient\footnote{\url{https://jardinage.lemonde.fr/dossier-820-sorgho-sorghum-bicolor-cereale-sans-gluten.html}}. Depuis
des siècles, les peuples d’Afrique et d’Asie utilisent ses graines
pour leur alimentation. Aujourd’hui, le sorgho est cultivé sur tous
les continents, sa domestication a vraisemblablement eu lieu il y a
plusieurs millénaires en Afrique et au Sud-est du Sahara. On note en
Afrique\,trois centres géographiques actifs dans la diversification du
sorgho cultivé: le centre ouest-africain qui a contribué à
l’établissement des sorghos de race guinea; le centre est-africain
riche en sorgho des races caudatum et durra; le centre sud-africain à
l’origine des sorghos de race kafir. Dès le troisième millénaire avant
J.-C., ces sorghos auraient gagné l’Asie: l’Inde et le Pakistan,
3\,000 ans avant J.-C., puis la Chine, un millénaire plus
tard. L’arrivée du sorgho en Europe se situerait vers 2\,000 ans avant
J.-C. Transporté en Amérique à l’époque des grandes découvertes au
XV\ieme{} siècle, le sorgho est véritablement diffusé qu’à partir du
XIX\ieme{} siècle, notamment aux
États-Unis\footnote{\url{https://www.gnis-pedagogie.org/sorgho-intro-caracteristiques-plante.html}}.

\subsection{Taxonomie}

Le sorgho, \emph{Sorghum bicolor (L.) Moench}, est une herbacée
annuelle de la famille des Poacées (ex-Graminées),de la sous famille
des Panicoïdeae, de la tribu des Andropogoneae et du genre
\emph{Sorghum}\,\cite{Doggett_1988}.  C’est une espèce monoïque
préférentiellement autogame. Le taux d’allogamie varie selon la race
considérée: très faible pour les variétés cléistogames qui subissent
une autopollinisation automatique et ainsi leurs fleurs ne s’ouvrent
qu’après l’anthèse (période terminale\,du développement de la fleur
depuis son épanouissement jusqu’au flétrissement). Ce taux
d'autopollinisation est de l’ordre de 5 à 7\,\% pour les variétés à
panicules compactes\,\cite{Doggett_1988}, et varie largement (20 à
29\%) pour les variétés à panicules lâches de la race botanique
Guinea \cite{Ollitrault_1987,Chantereau_1994}. De manière
schématique, la plante du sorgho se présente comme suit:


 
\begin{figure}%
  \begin{center}
    \includegraphics[width=16cm]{images/SchemaComposePage5}
  \end{center}
  \caption{Plante, graines --- Schéma et représentation en champ réel d'un plant de sorgho.}
   \label{fig-SchemaComposePage5}
\end{figure}

En champ réel les plants de sorgho se présentent ainsi (voir
Fig.~\ref{fig-SchemaComposePage5}
p.~\pageref{fig-SchemaComposePage5}).


Par ailleurs, \citeA[p.~18]{Chantereau_2013}, résument les différentes races du sorgho et
leurs caractéristiques d'identification (Tableau~\ref{tableau:Chantereau_2013}).
%\footnote{Google books}:


%%   %insérer un tableau
\begin{table}
  \begin{footnotesize}
    \begin{center}
      \begin{tabular}{lb{3cm}b{3cm}b{3cm}}
       \textbf{Race} & \textbf{Glumes}  & \textbf{Grains}  & \textbf{Panicules} \\ \hline
        Bicolor & Glumes longues recouvrant les 3/4 ou la totalité du grain  & Poids de 1000 grains de 15 à 25 g & Panicules lâches \\ \hline
        Guinea & Glumes généralement longues, ouvertes & Grains elliptiques, plus ou moins aplatis dorso-ventralement, de taille variable & Panicules lâches à semi-lâches, souvent longues à port retombant \\ \hline
        Caudatum & Glumes courtes adhérant au grain en le recouvrant partiellement & Grains dissymétriques de taille moyenne à grosse & Panicules compactes à semi-compactes souvent portées par un pédoncule crossé \\ \hline
        Durra & Glumes courtes adhérant au grain en le recouvrant partiellement & Grains plus ou moins sphériques, de taille variable mais le plus souvent gros à très gros & Panicules compactes à semi-compactes souvent portées par un pédoncule crossé \\ \hline
        Kafir & Glumes courtes adhérant au grain en le recouvrant partiellement & Grains elliptiques, de taille moyenne, poids de 1\,000 grains de 20 à 35 g & Panicules moyennement compactes, souvent de forme longue et cylindrique \\ 
      \end{tabular}
      \caption{Principaux caractères identitaires des races du sorgho \protect\cite{Chantereau_2013}.}
      \label{tableau:Chantereau_2013}
    \end{center}
  \end{footnotesize}
\end{table}


 
\newpage

%%   Tableau 1: Principaux caractères identitaires des races du sorgho.


L’inflorescence est une panicule de forme variable. Le grain est\,un
caryopse de couleur variable (blanche, rouge, brune et jaune), qui, à
maturité, est plus ou moins dégagé des
glumes\,\cite{SaintClair_1989,Chantereau_1991}. Tout comme le maïs et
la canne à sucre, le sorgho appartient à la tribu des
Andropogoneae. Les Andropogoneae sont en effet, une tribu de plantes
monocotylédones de la famille des Poaceae et de la sous-famille des
Panicoïdeae.

La figure 2 illustre la classification (cladogramme) phylogénétique
des Poaceae dont fait partie le sorgho \cite{Paquet_2005}.

%%%insérer une image


\begin{figure}%
  \begin{center}
    \includegraphics[width=10cm]{images/PhylogenyPoacea}
  \end{center}
\caption{Classification phylogénétique des Poacées \protect\cite{Paquet_2005}.}
\end{figure}


\subsection{Écologie}

\subsubsection{Photosensibilité}

Le sorgho est une plante de jours courts qui réagit de diverses façons
à la photopériode. À des latitudes élevées, certains cultivars
tropicaux ne fleurissent pas ou ne produisent pas de graines. Aux
États-Unis, en Australie et en Inde, on a noté l’existence de
cultivars moyennement sensibles à quasiment insensibles à la
photopériode\,\cite{BARRO_KONDOMBO_2010}.

\subsubsection{Conditions écologiques}

\paragraph{Conditions thermiques} Le sorgho tolère des températures de
tous niveaux. En effet, la sélection a permis de créer des variétés
cultivables en zones tempérées. Là où il ne gèle pas, le sorgho
continue à pousser et à produire de nouvelles feuilles vertes tant que
l’humidité du sol persiste. Il stoppe sa croissance sous les
8\,\degree{}C et meurt lorsque la température passe sous les
3\,\degree{}C\footnote{https://jardinage.ooreka.fr/plante/voir/2001/sorgho}. Il
est largement cultivé dans les régions tempérées et sous les tropiques
jusqu’à 2\,300 m d’altitude. La température optimale est de 25 à
31\,\degree{}C, mais des températures aussi faibles que 21\,\degree{}C
n’ont pas d’incidence grave sur la croissance et le rendement. Mais si
la température nocturne tombe en dessous de 12 à 15\,\degree{}C au
cours de la période de floraison, cela peut entraîner la stérilité. Le
sorgho est sensible au gel, mais moins que le maïs, et de légères
gelées nocturnes pendant la période de maturation provoquent peu de
dégâts\,\cite{BARRO_KONDOMBO_2010}.

\paragraph{Conditions hydriques} Le sorgho est une plante des milieux
tropicaux chauds et semi-arides qui sont trop secs pour les maïs
modernes, mais il existe aussi des maïs de zones sèches. Il est
particulièrement adapté à la sécheresse en raison d’un ensemble de
caractéristiques morphologiques et physiologiques, notamment un
système racinaire étendu, la pruine de ses feuilles qui limite ses
pertes en eau (figure~1), et une aptitude à interrompre sa croissance
pendant les périodes de sécheresse et à la reprendre une fois le
stress disparu. Le sorgho est une plante monocotylédone de la tribu
des Andropogoneae.Toutes les espèces de cette tribu présentent une
photosynthèse en $\text{C}_4$. La photosynthèse en $\text{C}_4$ se
réalise chez de nombreuses graminées tropicales à savoir le maïs, la
canne à sucre, le mil,le sorgho poussant dans des conditions
désertiques ou sur des sols salés. Elle se traduit par un type de
fixation du gaz carbonique où une enzyme cytoplasmique
phospho-énolpyruvate carboxylase catalyse une réaction avec comme
accepteur de $\text{CO}_2$ le phosphoénolpyruvate (PEP)\,: $PEP +
\text{CO}_2 \longrightarrow \text{Oxaloacétate}$. On parle de
photosynthèse en $\text{C}_4$ puisqu'on aboutit à la formation de
molécules à 4 atomes de carbone. À l’opposé de certaines graminées
dites plantes en $\text{C}_3$ qui réalisent la fixation du carbone en
$\text{C}_3$, la plante du sorgho dite plante en $\text{C}_4$, a un
avantage compétitif dans ce sens où lorsqu’elle est soumise à la
sécheresse, à la chaleur et à un faible taux d’azote ($N_2$) ou de
dioxyde de carbone ($\text{CO}_2$) et lorsqu’elle est cultivée par
exemple dans un environnement à 30\,\degree{}C, elle perd moins de
molécules d’eau (les graminées en $\text{C}_3$ perdent environ 833
molécules d’eau par molécule de $\text{CO}_2$ fixée tandis que le
sorgho en perd seulement 277)\,\cite{sage1998c4}. Ceci offre donc un
avantage aux plantes en $\text{C}_4$ (sorgho) dans les environnements
arides (zone sahélienne au Burkina Faso)\,\cite{sage1998c4}. Des
 précipitations de 500 à 800 mm également réparties pendant la saison
 de production conviennent généralement aux cultivars qui mûrissent
 en 3 à 4 mois. Le sorgho tolère un certain niveau d’asphyxie
 racinaire et on peut le faire pousser dans des zones à fortes
 précipitations\,\cite{BARRO_KONDOMBO_2010}.

 
\paragraph{Conditions édaphiques}
Le sorgho est bien adapté aux vertisols lourds. En pédologie un
vertisol est un sol riche en argile du type 2/1 c’est-à-dire contenant
une couche d’oxyde d’aluminium enserrée par deux couches de tétraèdres
de
silice\footnote{\url{https://www.memoireonline.com/06/12/5956/Analyse-des-determinants-de-la-faible-productivite-du-mas-a-Agadjaligbo-dans-la-commune-de-Zogbo.html}}
que l’on trouve couramment dans les tropiques, où sa tolérance à
l’asphyxie racinaire est souvent nécessaire, mais les sols sableux
légers lui conviennent tout autant. C’est toutefois sur les limons et
les limons sableux que sa culture réussit le mieux. La fourchette de
pH du sol supportée par le sorgho est de 5.0 à 8.5, et il tolère
davantage la salinité que le maïs (6 à
7.5)\footnote{\url{http://www.bioactualites.ch/fileadmin/documents/bafr/production-vegetale/grandes-cultures/4.5.11-73_Mais.pdf}}. Il
est adapté aux sols pauvres et peut produire du grain sur des sols où
beaucoup d’autres cultures échoueraients\,\cite{BARRO_KONDOMBO_2010}.
 
 
\subsection{Morphologie et biologie} La plante du sorgho
(\emph{Sorghum bicolor}) comprend une tige principale accompagnée de
talles issues du développement de bourgeons adventifs sur le collet du
maître brin. La hauteur de la plante à maturité varie beaucoup (de
50\,cm à plus de 5\,m). En fonction des cultivars et de leur
situation, les feuilles (alternes, longues, retombantes, vert clair ou
vert foncé) portées par les tiges varient en nombre (de quelques
unités à plus de 30): figure 1\,\cite{BARRO_KONDOMBO_2010}. Le
\emph{S.bicolor} est principalement autogame, mais une pollinisation
croisée par le vent peut se produire dans certaines conditions, à plus
de 60\% selon le génotype, et en moyenne environ
6\,\%\,\cite{Ellstrand_1983,
  House85,Pedersen_1998,Schertz_1980}. Puisque l’espèce se reproduit à
la fois par autopollinisation et par pollinisation croisée, la plupart
des races locales de sorgho cultivées par les agriculteurs de
subsistance sont constituées de mélanges de lignées pures et de
lignées partiellement pures\,\cite{SINGH_1997}. Le degré d’allogamie
varie notamment en fonction du type de panicule du cultivar;
généralement, la pollinisation croisée est plus élevée dans le cas des
sorghos herbeux à panicule lâche que dans celui des sorghos cultivés à
panicule compacte. Selon certaines estimations, le taux d’allogamie
chez le sorgho cultivé en plein champ varie de 5 à plus de
40\,\%\,\cite{Barnaud_2008, DJE_2004, Doggett_1988,
  Ellstrand_1983,Schmidt_2006}. Plusieurs espèces de pollinisateurs
ont été observées consécutivement visitant des fleurs de sorgho
cultivé\,\cite{Immelman_2000, Schmidt_2006}. Lors de la collecte des
insectes, des grains de pollen de sorgho ont été trouvés sur chacun de
ceux-ci. Cependant, il n’a pas été déterminé si le déplacement des
insectes occasionnait une pollinisation croisée. D’autres études
doivent être réalisées pour déterminer l’importance de la
pollinisation par les insectes chez le \emph{S. bicolor}. La floraison
et la pollinisation du \emph{S.bicolor} sont décrites
dans\,\citeauthor{House85}\,\citeyear{House85},
\citeauthor{SINGH_1997} \,\citeyear{SINGH_1997} et
\citeauthor{SRINIVASA_2013}\,\citeyear{SRINIVASA_2013}. L’inflorescence
commence à se former 30 à 40 jours après la germination. Le sorgho
cultivé fleurit généralement 55 à 70 jours après la germination en
climats chauds, mais, selon le génotype, la plante peut fleurir 30 à
100 jours après la germination. Le temps humide et frais peut retarder
la floraison. Les fleurs commencent à s’ouvrir deux jours après que
l’inflorescence a émergé de la gaine. Les épillets (subdivisions de
l’inflorescence qui comportent plusieurs fleurs) sessiles, situés au
sommet de l’inflorescence, sont les premiers à fleurir, puis la
floraison se poursuit vers le bas de l’inflorescence durant 4 ou 5
jours. Chaque panicule peut comprendre jusqu’à 6\,000
fleurons\,\cite{QUINBYKARPER_1947}. La floraison ne survient pas au
même moment chez toutes les inflorescences dans un champ, de sorte que
le pollen est généralement présent durant 10 à 15 jours. Le moment de
la floraison varie en fonction du génotype et du climat, mais celle-ci
se produit généralement du milieu de la nuit au milieu de la matinée
et atteint son maximum vers le lever du soleil. Le gonflement des
lodicules facilite l’ouverture des fleurs. Lorsque les stigmates
deviennent visibles, le filet des étamines s’allonge, et les anthères
deviennent pendantes. Une fois que les anthères sont sèches, le pore
apical s’ouvre et le pollen est libéré. La majeure partie du pollen
d’une inflorescence fertilise les ovules de la même inflorescence. La
pollinisation croisée est possible lorsque le pollen est transporté
dans les airs. Le stigmate est pollinisé avant que les anthères
émergent des épillets. Les grains de pollen sont transportés jusqu’au
stigmate et germent. Un tube pollinique se forme, et le grain de
pollen divisé en deux noyaux descend dans le style pour aller
fertiliser l’ovule. Un noyau spermatique fertilise l’ovule et produit
un embryon $2n$, et l’autre noyau fusionne avec les noyaux polaires
pour produire un albumen $3n$. Après la pollinisation, les glumes se
referment, et les anthères et stigmates vides en dépassent
généralement. Certaines variétés à glumes longues sont cléistogames
(les fleurons ne s’ouvrent pas pour la fertilisation). Les stigmates
non fécondés demeurent réceptifs jusqu’à 16\,jours. Après la
fertilisation, la différenciation des organes se déroule sur environ
douze jours. Les graines passent par trois stades de développement
---\,laiteux, pâteux mou et pâteux dur\,--- et parviennent à maturité
en environ 30\,jours. Le \emph{S.~bicolor} se reproduit par ses
graines~\footnote{\url{http://www.inspection.gc.ca/vegetaux/vegetaux-a-caracteres-nouveaux/demandeurs/directive-94-08/documents-sur-la-biologie/sorghum-bicolor-l-moench/fra/1490144063487/1490144119854}}.


\subsection{Intérêt et utilisation du sorgho}

\emph{Sorghum bicolor} (L.) Moench est une céréale importante,
particulièrement pour les zones chaudes à pluviométrie réduite de la
zone tropicale. Le sorgho occupe le cinquième rang des plus
importantes céréales dans le monde, qu’il s’agisse du volume de la
production ou des superficies cultivées\,\cite{FAOICRISAT_1997}. En
Afrique subsaharienne, le sorgho est la deuxième céréale en importance
après le maïs. Le Nigeria en est le premier producteur de la région
avec 7\,\% de la production mondiale\,\cite{FAO_1995}. Le sorgho est
la principale céréale cultivée au Burkina Faso, avec plus d’un million
et demi
d’hectare\,\footnote{\url{https://www.cirad.fr/nos-recherches/resultats-de-recherche/2016/la-selection-participative-du-sorgho-au-burkina-faso-creer-de-nouvelles-varietes-avec-et-pour-les-paysans}}. C’est
la plus importante culture vivrière dans les régions tropicales
semi-arides d’Afrique. Selon Dahlberg et al, le sorgho est l’aliment
de base de 500 millions de personnes dans plus de 30 pays de la zone
tropicale semi-aride
\,\footnote{\url{http://www.memoireonline.com/01/14/8569/m_Contribution--l-etude-des-contraintes-de-stockage-des-cereales-mil-mas-sorgho-en-zone-sud-s8.html}}.
Sa culture y est pratiquée partout en saison des pluies, là où les
précipitations sont supérieures aux valeurs 400 à 500 mm. La
superficie consacrée à la culture du sorgho est passée de 1\,016\,275
ha en 2\,000 à 1\,619\,590 ha en 2007\,\cite{FAO_2007} et couvrait
en 2016 1,8 millions d’hectares
\,\footnote{\url{http://www.commodafrica.com/14-11-2016-la-production-de-sorgho-en-afrique-progresseraitr-de-23-en-20161}}. Dans
les régions tropicales, le sorgho est essentiellement cultivé pour son
grain destiné d’abord à l’alimentation humaine. Le grain peut être
consommé entier ou décortiqué et réduit en poudre pour faire la
bouillie, le tô (nom d’un des plats locaux cuisiné sous forme de pâte
dans plusieurs pays d’Afrique subsaharienne dont le Burkina Faso), du
couscous et des beignets. Le grain peut être fermenté pour donner des
boissons alcoolisées: bière (dolo) ou du vin de
sorgho\,\cite{Memento_1991}. Les résidus de récolte, soit l’ensemble
des tiges, feuilles et panicules égrenées, représentent pour
l’agriculteur une importante source de fourrage pour l’alimentation de
son bétail\,\cite{Chantereau_1991}. Les graines (entiers) sont
fournies directement aux volailles. Les tiges sont employées pour la
confection des nattes, dans la construction des maisons (matériaux de
construction pour réaliser des palissades) et des enclos ou comme
combustible\,\cite{SaintClair_1989}. D’autres utilisations du sorgho
sont à signaler: la paille (feuilles et tiges) sert comme
fourrage. Les tiges de certains sorghos bicolor sont consommées en
frais comme la canne à sucre. Les cendres servent à la préparation de
la potasse alimentaire. La moelle de sorgho est utilisée comme support
pour les coupes anatomiques. Certains sorghos à forte coloration
tannique servent dans la teinture du cuir. Les extraits de composés
phénoliques servent en cosmétique pour le bronzage. Si dans les
principales régions productrices d’Afrique et d’Asie, plus de 70\,\% du
sorgho sont consommés par l’Homme, en Amérique du Nord, Amérique
centrale, Amérique du Sud et Océanie par contre la plus grande partie
de la production sert à l’alimentation
animales\,\cite{BARRO_KONDOMBO_2010}. La culture a également une
vocation industrielle orientée sur la production de la pâte à papier,
la production du fuel,
etc\,\footnote{\url{http://www.memoireonline.com/01/16/9362/m_Etude-de-la-diversite-agro-morphologique-du-sorgho-et-identification-de-cultivars-tolerants-au-str11.html}}.

En novembre 2016, l’USDA (Unitites States Department of Agriculture) a résumé les situations africaine et mondiale de production du sorgho (figure 3):

%\newpage

%\begin{figure}%
 % \begin{center}
 % \includegraphics[width=14cm]{images/WorldSorghumStatisticsUsda}
 %\end{center}
 %\caption{Tableau 1: Situations africaines et mondiale de production du sorgho, selon l’USDA;p.:
   % la production réelle et est.:les estimations}
%\end{figure}

\begin{table}
  \begin{center}
    \begin{tabular}{|c|p{4.5cm}|c|p{4.5cm}|c|p{4.5cm}|c|p{4.5cm}|}
      \hline
      \multirow{3}{*}{Localisation} & \textbf{Superficies}      & \textbf{Rendements}  & \textbf{Production}          \\ \cline{2-3}
      & Millions ha                & T/ha                 & millions de tonnes          \\ \cline{2-3}
      & 2015/16\,(p.) 2016/17\,(est.) & 2015/16\,(p.)  2016/17\,(est.) & 2015/16\,(p.) 2016/17\,(est.)        \\ \hline
      \multirow{1}{*}{\textbf{Monde}} & 42.54  \, 42.49 & 1.41  \, 1.51 & 60.16 \, 64.20 \\ \hline
      \multirow{1}{*}{Afrique} & 24.09 \, 24.41 & 1.35 \, 1.49 & 19.39 \, 23.9 \\ \hline
      \multirow{1}{*}{Nigeria} & 5.30  \, 5.30 & 1.16  \, 1.23 & 6.15 \, 6.50 \\ \hline
      \multirow{1}{*}{Burkina} & 1.80  \, 1.80 & 0.80  \, 1.06 & 1.44 \, 1.90 \\ \hline
      \multirow{1}{*}{Mali} & 1.30  \, 1.30 & 1  \, 1  & 1.30  \, 1.30  \\ \hline
      \multirow{1}{*}{Niger} & 3.50  \, 3.50 & 0.55  \, 0.37  & 1.92  \, 1.30  \\ \hline
      \multirow{1}{*}{Ghana} & 0.25  \, 0.25 & 1.05  \, 1.20  & 0.26  \, 0.30  \\ \hline
      \multirow{1}{*}{\textbf{Sous total Afrique Ouest}} & 12.15  \, 12.15 & 0.912 \, 0.972  & 11.07  \, 11.3  \\ \hline
      \multirow{1}{*}{Ethiopie} & 1.50  \, 1.80 & 1.73  \, 2.06  & 2.60  \, 3.70  \\ \hline
      \multirow{1}{*}{Soudan} & 8  \, 8 & 0.30  \, 0.69  & 2.39  \, 5.50  \\ \hline
      \multirow{1}{*}{Cameroun} & 0.80  \, 0.80 & 1.73  \, 2.06  & 2.60  \, 3.70  \\ \hline
      \multirow{1}{*}{Tanzanie} & 0.80  \, 0.80 & 1.03  \, 1  & 0.82  \, 0.80  \\ \hline
      \multirow{1}{*}{Egypte} & 0.14  \, 0.14 & 5.36  \, 5.36  & 0.75  \, 0.75  \\ \hline
      \multirow{1}{*}{Ouganda} & 0.35  \, 0.35 & 0.91  \, 0.91  & 0.32  \, 0.32  \\ \hline
      \multirow{1}{*}{Mozambique} & 0.30  \, 0.30 & 0.72  \, 0.67  & 0.22  \, 0.20  \\ \hline
      \multirow{1}{*}{Afrique du Sud} & 0.05  \, 0.07 & 1.51  \, 2.50  & 0.07  \, 0.18  \\ \hline
    \end{tabular}
    \caption{Situations africaines et mondiale de production du
      sorgho, selon l'USDA avec la production réelle (p.) et les
      estimations (est.)\protect\footnote{\protect\url{http://www.commodafrica.com/14-11-2016-la-production-de-sorgho-en-afrique-progresseraitr-de-23-en-20161}}.}
  \end{center}
\end{table}


%\begin{table}
 % \begin{center}
  %  \begin{tabular}{|c|p{4.5cm}|c|p{4.5cm}|c|p{4.5cm}|c|p{4.5cm}|}
   %   \hline
    %  \multirow{3}{*}{Localisation} & \textbf{Superficies}      & \textbf{Rendements}  & \textbf{Production}          \\ \cline{2-3}
     % & multicolumn{2}{*}{millions ha}                 & T/ha                 & millions de tonnes          \\ \cline{2-3}
      %& 2015/16\,(p.) 2016/17\,(est.) & 2015/16\,(p.)  2016/17\,(est.) & 2015/16\,(p.) 2016/17\,(est.)        \\ \hline
      %\multirow{1}{*}{\textbf{Monde}} & 42.54  \, 42.49 & 1.41  \, 1.51 & 60.16 \, 64.20 \\ \hline
      %\multirow{1}{*}{Afrique} & 24.09 \, 24.41 & 1.35 \, 1.49 & 19.39 \, 23.9 \\ \hline
      %\multirow{1}{*}{Nigeria} & 5.30  \, 5.30 & 1.16  \, 1.23 & 6.15 \, 6.50 \\ \hline
     % \multirow{1}{*}{Burkina} & 1.80  \, 1.80 & 0.80  \, 1.06 & 1.44 \, 1.90 \\ \hline
     % \multirow{1}{*}{Mali} & 1.30  \, 1.30 & 1  \, 1  & 1.30  \, 1.30  \\ \hline
     % \multirow{1}{*}{Niger} & 3.50  \, 3.50 & 0.55  \, 0.37  & 1.92  \, 1.30  \\ \hline
     % \multirow{1}{*}{Ghana} & 0.25  \, 0.25 & 1.05  \, 1.20  & 0.26  \, 0.30  \\ \hline
    %  \multirow{1}{*}{\textbf{Sous total Afrique Ouest}} & 12.15  \, 12.15 & 0.912 \, 0.972  & 11.07  \, 11.3  \\ \hline
     % \multirow{1}{*}{Ethiopie} & 1.50  \, 1.80 & 1.73  \, 2.06  & 2.60  \, 3.70  \\ \hline
     % \multirow{1}{*}{Soudan} & 8  \, 8 & 0.30  \, 0.69  & 2.39  \, 5.50  \\ \hline
     % \multirow{1}{*}{Cameroun} & 0.80  \, 0.80 & 1.73  \, 2.06  & 2.60  \, 3.70  \\ \hline
     % \multirow{1}{*}{Tanzanie} & 0.80  \, 0.80 & 1.03  \, 1  & 0.82  \, 0.80  \\ \hline
     % \multirow{1}{*}{Egypte} & 0.14  \, 0.14 & 5.36  \, 5.36  & 0.75  \, 0.75  \\ \hline
     % \multirow{1}{*}{Ouganda} & 0.35  \, 0.35 & 0.91  \, 0.91  & 0.32  \, 0.32  \\ \hline
     % \multirow{1}{*}{Mozambique} & 0.30  \, 0.30 & 0.72  \, 0.67  & 0.22  \, 0.20  \\ \hline
     % \multirow{1}{*}{Afrique du Sud} & 0.05  \, 0.07 & 1.51  \, 2.50  & 0.07  \, 0.18  \\ \hline
   % \end{tabular}
   % \caption{Situations africaines et mondiale de production du
   %   sorgho, selon l'USDA avec la production réelle (p.) et les
    %  estimations (est.)\protect\footnote{\protect\url{http://www.commodafrica.com/14-11-2016-la-production-de-sorgho-en-afrique-progresseraitr-de-23-en-20161}}.}
 % \end{center}
%\end{table}


\section{Le niébé: \emph{Vigna unguiculata} (L.) Walp.}
%%   %insérer l'image ( figure 5)
\begin{figure}%
  \begin{center}
   \includegraphics[width=12cm]{images/graines_niebe}
  \end{center}
\caption{Quelques variétés de niébé au Burkina Faso}
\end{figure}

\newpage

%%   Figure 5:Quelques variétés de niébé (graines et plantes) observées au Burkina Faso.


\subsection{Origine et diffusion}

Le niébé,\emph{V. unguiculata} est une légumineuse annuelle dont le
centre d’origine était incertain avant les études de
Faris\,\cite{FARIS_1963,FARIS_1965}. \citeauthor{Piper_1913}\,\citeyear{Piper_1913},
a donné une double origine au niébé: l’Inde et
l’Afrique.\cite{FARIS_1963,FARIS_1965} après des études qui
se sont reposées sur une description cytologique et morphologique des
formes sauvages et cultivées du niébé montre que l’Afrique de l’Ouest
et plus probablement le Nigeria est le centre d’origine du niébé. Sa
diffusion vers l’Asie, en parallèle avec celle du sorgho, date de
2\,300 av J-C. Le niébé est introduit en Europe vers 300 av J-C où il
reste une culture mineure dans la partie méridionale. Les Espagnols et
les Portugais l’exportent au XVII\ieme{} siècle vers le nouveau
monde. D’autres cultivars sont transportés directement de l’Afrique
vers l’Amérique latine avec le trafic des esclaves. Le niébé atteint
le Sud des États-Unis au début du XIX\ieme{}
siècle\,\cite{Sawadogo_2009}.

\subsection{Taxonomie}

Le niébé a été décrit par Linné, à partir d’une forme cultivée
provenant des Antilles, sous le nom de Dolichos unguiculatus, qui
deviendra Vigna unguiculta\,\cite{Pasquet_1997}. Vigna unguiculta
inclut des formes cultivées et des formes sauvages. Les formes
cultivées se distinguent des formes sauvages par des gousses
indéhiscentes, des graines et des gousses de taille plus importante
\,\cite{Lush_1981}. Selon\,\citeauthor{Vanderborght_2001}\,\citeyear{Vanderborght_2001},
les formes cultivées sont regroupées dans la sous-espèce unguiculta,
laquelle est subdivisée en quatre cultigroupes: le cultigroupe
Unguiculta (anciennement \emph{ V. sinensis (L.) Savi ex Hassk.}),
forme couramment cultivée et plus importante en Afrique; le
cultigroupe Biflora (anciennement V. unguiculata subsp. Cylindrica
(L.) Verdcourt), à petites gousses érigées, cultivé principalement en
Asie; le cultigroupe Sesquipedalis (anciennement \emph{V.unguiculata
  var. Sesquipedalis (L.) Ohashi}), à gousses très longues et
pendantes; le cultigroupe Texfilis avec de long pédoncule est présent
en Afrique de l’Ouest. Le niébé est une dicotylédone de l’ordre des
Fabales,de la famille des Fabaceae, de la sous famille Faboideae,de la
tribu Phaseoleae,de la sous tribu des Phaseolinae, du genre
Vigna\,\cite{Verdcourt_1970, Marechal_1978}.

\subsection{Écologie}

\subsubsection{Photosensibilité}

Ce caractère a été largement étudié, en particulier
par\,\citeauthor{Steele_1972}\,\citeyear{Steele_1972}. On distingue
trois groupes, \textbf{le premier groupe}, photo-indépendant tardif,
comprend des génotypes indifférents à la
photopériode\,\footnote{\url{https://www.doc-developpement-durable.org/file/Culture-plantes-alimentaires/FICHES_PLANTES/niebe/Cowpea_niébé_culture_french.pdf}}. La
croissance est indéterminée et le port quelquefois érigé mais le plus
souvent volubile. Ces génotypes sont généralement tardifs et ont une
floraison échelonnée à partir de nœuds éloignés au cours de la saison
culturale. On trouve ces cultivars dans les zones les plus proches de
l’équateur comme les savanes guinéennes humides, où ils sont cultivés
surtout en première saison humide. \textbf{Le deuxième groupe},
photo-indépendant précoce, est constitué des génotypes également
indifférents à la photopériode. Ces génotypes fleurissent précocement
à partir des premiers nœuds de la tige principale et donnent une
production groupée, souvent récoltable au bout de deux mois. Ces
variétés sont cultivées dans les zones de latitude élevée, en Inde,
dans le bassin méditerranéen et aux États-Unis. \textbf{Le troisième
  groupe}, photosensible, regroupe des génotypes sensibles à la
photopériode\,\cite{Steele_1972}. Le port est généralement rampant et
nettement moins volubile que chez les cultivars du premier groupe. Ce
groupe englobe la plupart des cultivars traditionnels de la région
soudano-sahélienne cultivés en association avec le sorgho et le
mil\,\cite{Doggett_1988}. Il est à noter que les deux derniers groupes
photo-indépendants précoce et photosensible sont assez
proches. Cultivés en jours très courts, ils sont indiscernables et les
cultivars photosensibles présentent alors un port érigé et fleurissent
dès les premiers nœuds. En revanche, les deux premiers groupes
photo-indépendant tardif et photo-indépendant précoce sont bien
distincts. Cultivés en jours longs, ils sont tardifs mais leurs ports,
rampant pour l’un et volubile pour l’autre, apparaissent très
différents. De plus, le groupe photo-indépendant précoce et le groupe
photosensible ont relativement peu d’ovules par rapport au groupe
photo-indépendant tardif. Ainsi, ce n’est pas le photopériodisme qui
permet de séparer les cultivars de niébé en deux grands groupes
physiologiques comme le supposait
\,\citeauthor{Steele_1972}\,\citeyear{Steele_1972}, mais l’aptitude à
fleurir rapidement dès les premiers nœuds de la tige principale dans
des conditions très inductives de jours
courts\footnote{\url{https://www.doc-developpement-durable.org/file/Culture-plantes-alimentaires/FICHES_PLANTES/niebe/Cowpea_niébé_culture_french.pdf}}.

\subsubsection{Conditions écologiques}

\textbf{Conditions édaphiques} Le niébé se cultive sur les sols
sableux à argileux. Il ne supporte pas l’engorgement et l’acidité du
sol\,\cite{Doggett_1988}. Le niébé croît bien à des pH de 4,5 à 9,0 et
réussit à fixer l’azote dans des sols possédants moins de 2\% de
matière organique et plus de 80\% de
sable\,\cite{SINGH_1997}. \textbf{Conditions thermiques} Le niébé est
une culture très bien adaptée aux régions arides et semi-arides. C’est
une légumineuse cultivée dans les régions tropicales et
subtropicales\,\cite{Doggett_1988}. \textbf{Conditions hydriques} Le
niébé est une plante ayant une certaine adaptation à la sécheresse. Sa
culture est effectuée entre les isohyètes 300\,mm à
1\,500\,mm\,\cite{Doggett_1988}.

\subsection{Morphologie et biologie}

  
\paragraph{L’appareil végétatif:}

\textbf{La tige} la tige du niébé est cylindrique, volubile, quelques fois
glabre et creuse. Elle définit le port de la plante qui peut être
érigé, semi-érigé, buissonnant, ou rampant. Chaque nœud de la tige
porte deux stipules et trois bourgeons axillaires. \textbf{Les feuilles} la
première paire de feuille est opposée et monofoliée. Les secondes
feuilles sont alternes et trifoliées comprenant deux folioles opposées
et une foliole terminale.\textbf{Les racines} le système racinaire est
pivotant avec de nombreuses ramifications, ce qui confère au niébé une
certaine tolérance à la sécheresse. Les racines portent des nodosités
de bactéries fixatrices d’azote atmosphérique\,\cite{Doggett_1988}.



\paragraph{L’appareil reproducteur:}

\textbf{L’inflorescence} est un racème axillaire. Le pédoncule a
une longueur variable, au bout duquel se trouve le rachis. La
coloration des fleurs varie du blanc au violet en fonction de la
concentration d’anthocyanine. Le niébé est une plante
autogame\,\cite{Fery}. Le cycle des variétés est déterminé au stade
50\% de floraison\,\cite{Drabo_1981}. \textbf{Les fruits} le fruit du niébé
est une gousse pendante ou dressée avec des formes linéaire, spiralée,
ou enroulée. \textbf{La gousse} peut être entièrement pourpre, pigmentée sur
les valves à son extrémité, marbrée ou dépourvue de
pigments. \textbf{Les graines} la graine du niébé comporte un tégument
qui peut être ridé ou lisse. Elle est de couleur, de taille, et de
forme variables. La graine est riche en protéines et en
carbohydrates\,\cite{Doggett_1988}.

\begin{figure}%
  \begin{center}
   \includegraphics[width=18cm]{images/SchemaDescriptifNiebe}
  \end{center}
  \caption{Schéma illustrant les composantes du niébé\protect\footnote{\protect\url{http://jardinonssolvivant.fr/WordPress/wp-content/uploads/2018/01/figure-1-1024x445.jpg}}.}
\end{figure}


\subsection{Intérêt et utilisation du niébé}

Le terme niébé est un mot wolof désignant une plante légumineuse,
\emph{Vigna unguiculata (L.) Walp}. Herbacée africaine, sa culture
présente des retombées économiques, nutritionnelles et agronomiques
considérables entre autre dans de nombreux pays d'Afrique de l'Ouest y
compris les pays sahéliens tels que Burkina Faso (les zones
centre-nord et nord qui sont de fortes productrices de niébé).  Le
niébé est une source de devises pour les pays producteurs. Les graines
du niébé alimentent les échanges économiques au niveau régional et
sous régional. Selon \citeauthor{Langyintuo_2003},
\,\citeyear{Langyintuo_2003}\,le Niger, le Burkina Faso, le Bénin, le
Mali, le Cameroun, le Tchad, et le Sénégal sont les principaux pays
exportateurs du niébé; tandis que le Nigeria, le Ghana, le Togo, la
Côte d’ivoire, le Gabon, et la Mauritanie sont les pays
importateurs. En plus du commerce des graines du niébé, le fourrage
est aussi commercialisé et utilisé dans l’alimentation des
animaux\,\cite{Langyintuo_2003}. En Afrique occidentale et centrale,
le commerce du fourrage du niébé permet une augmentation de 25\,\% du
revenu annuel des paysans\,\cite{Quin_1997}. Sur le plan nutritionnel:
les jeunes feuilles, les gousses immatures, et les graines sont
utilisées dans l’alimentation humaine. La valeur nutritionnelle des
graines est élevée avec en moyenne 23 à 25\,\% de protéines et 50 à
67\,\% d’amidon, ce qui confère au niébé un rôle important dans la
lutte contre la déficience protéique chez les
enfants\,\cite{Quin_1997}. Source de protéines moins coûteuse que
celle d’origine animale (viande, poisson, œuf), le niébé peut
contribuer de manière significative à la solution du problème de
déficit protéique souvent constaté en Afrique. De plus, les fanes de
niébé sont un aliment apprécié des animaux
domestiques\,\cite{BAMBARA_2008}. La graine du niébé est riche en
lysine mais déficiente en acide aminé soufré. Sur le plan agronomique:
la culture du niébé permet un enrichissement du sol en azote par
l’intermédiaire de bactéries fixatrices d’azote atmosphérique
(rhizobium). \citeauthor{Quin_1997}\citeyear{Quin_1997}, montre qu’un hectare de niébé fixe 40 à 80
Kg d’azote nitrés dans le sol. De par sa croissance rapide, le niébé
assure une couverture du sol,le protégeant ainsi contre l’érosion et
contre l’envahissement des adventices\,\cite{Sawadogo_2009}.

\section{Les sols en Afrique subsaharienne: focus sur le Burkina Faso}

\paragraph{Généralités}

Le sol constitue le réservoir où les plantes puisent l’eau et les
éléments minéraux nécessaires à leurs besoins. La capacité de ce
réservoir dépend non seulement des caractéristiques du sol, mais aussi
de la profondeur exploitable par les racine. La texture et surtout la
structure du sol jouent un rôle important dans la dynamique de
l’enracinement des cultures\,\cite{Chopart_1980}. La structure détermine
aussi la masse volumique sèche (densité apparente) du sol et par
conséquent la porosité. Les propriétés physiques du sol déterminent
également ses caractéristiques hydrodynamiques, notamment la
perméabilité, la capacité au champ, le point de flétrissement
permanent et la réserve en eau utile
\footnote{\url{http://hydrologie.org/redbooks/a199/iahs_199_0217.pdf}}.

%  %inserer une image

\begin{figure}%
 \begin{center}
   \includegraphics[width=12cm]{images/cartepedobf}
  \end{center}
  \caption{Carte des sols du Burkina Faso\,(Haute Volta)\protect\footnote{\protect\url{https://esdac.jrc.ec.europa.eu/ESDB_Archive/EuDASM/Africa/maps/afr_cpr_ha(es).htm}}.}
\end{figure}



Les sols du Burkina ont été étudiés par plusieurs institutions de
recherche et organismes de développement. En 1969, l'Institut de
Recherche Scientifique pour le Développement en Coopération (ORSTOM)
avait réalisé une première couverture du territoire et dressé une
carte pédologique qui constitue toujours aujourd'hui la base de la
plupart des études pédologiques menées dans le pays\,\cite{BUNASOLS_2004}.
Les pédologues ont classé l'ensemble des sols du pays
en huit groupes principaux, qui a été proposé par le Bureau National
des Sols (Bunasols) en 1985 sur la base d'autres études antérieures
\,\cite{PERON_1975}. De façon générale, les sols du Burkina ont un
faible niveau de teneur en éléments fertilisants, notamment en
phosphore et en azote. Leur profondeur est généralement limitée par une
cuirasse qui affleure même en surface en certains endroits. Le niveau
de la réserve en eau varie avec la toposéquence. Généralement faible
en haut de pente (inférieure à 60 mm/mètre de sol), la réserve utile
peut dépasser 150 mm/mètre dans les sols alluvionnaires le long des
cours d'eau. C'est le cas par exemple des sols de la plaine irriguée
du Sourou\,\cite{SOMENICOU_1983}. Ces sols subissent de façon très accrue
le ruissellement et l'érosion hydrique\,\cite{Roose_2004}
\footnote{\url{http://hydrologie.org/redbooks/a199/iahs_199_0217.pdf}}. Les
8 principaux types de sols définis grâce aux travaux de l'ORSTOM au
Burkina sont: les sols ferrugineux lessivés, les sols peu évolués
d’érosion, les sols bruns eutrophes, les vertisols, les sols
ferrallitiques, les sols halomorphes, les sols hydromorphes et les
sols minéraux bruts. Les deux premiers types de sols occupent plus des
deux tiers du pays. Les sols ferrugineux lessivés couvrent les plus
grandes étendues. Ils sont localisés essentiellement dans la partie
méridionale de la pénéplaine précambrienne, au sud du XIII\ieme\,
parallèle. Ce sont des sols à texture variable, généralement à
tendance sableuse dans les horizons de surface et argileuse dans les
horizons plus profonds (au delà de 40\,cm). Ils ont un régime hydrique
imparfait, en rapport avec de mauvaises propriétés physiques (porosité
et perméabilité). Ils ont tous une faible capacité d’échange
cationique. Ils sont régulièrement associés à des sols
gravillonnaires. Les sols peu évolués d’érosion sont plutôt situés
dans la moitié nord du pays. Ils sont installés sur des granites et
des migmatites dont ils dérivent. Ils présentent un horizon sableux en
surface (15 à 20\,cm) et un horizon argileux au-delà. La compacité et
l’imperméabilité de ce second horizon jouent un rôle néfaste pour
l’alimentation hydrique et l’enracinement. Les sols hydromorphes sont
installés sur des alluvions fluviatiles ou sur des matériaux
d’altération fins. De faible drainage, ils s’engorgent régulièrement
en saison des pluies. Ils sont surtout développés dans l’ouest du pays
et s’alignent avec le réseau hydrographique majeur: vallées du
Mouhoun, du Nazinon et du Nakambé. Les sols bruns eutrophes sont
caractérisés par une fraction argileuse importante. La présence
d’argile gonflante leur confère une forte capacité d’échange et un
taux de saturation élevé. Ce sont des sols généralement bien
drainés. Leur structure de surface est variable, de grumeleuse à
prismatique. C’est cette propriété qui règle leur fertilité. Ils sont
répartis sur l’ensemble du territoire, par tâches de faible
étendue. Les vertisols possèdent la même parenté texturale que les
sols bruns. De fait, ce sont des sols beaucoup moins drainés. Ils sont
particulièrement développés dans le sud-est et le centre-ouest (vallée
du Sourou). Les sols minéraux bruts sont des sols de faible profondeur
installés sur la roche-mère ou sur des horizons cuirassés. Ce sont des
sols pauvres. La végétation qu’ils portent est tantôt clairsemée ou au
contraire dense à cause de leur faible aptitude agricole qui les met à
l’abri de toute intervention humaine. Les sols halomorphes ou salés
sont installés au nord du pays. De texture variée, ces sols ont une
structure franchement dégradée. Ce sont des sols pauvres qui
supportent des steppes arbustives extrêmement lâches. Les sols
ferrallitiques sont localisés dans le sud-ouest du pays où ils
occupent une faible surface. Leur profil s’apparente à celui des sols
ferrugineux, mais leurs propriétés physiques et chimiques les
différencient nettement. Ils constituent de bons supports pour les
cultures et pour la végétation naturelle dominée par les savanes
arborées. Les sols de notre échantillonnage ont été collectés sur des
parcelles situées dans la zone soudano-sahélienne du Burkina-faso
(Boussouma, Kaya, Korsimoro etc). Trois types de sols parmi ceux
sus-cités sont rencontrés dans cette zone d'étude: les sols
ferrugineux tropicaux lessivés, les sols hydromorphes peu humifères à
pseudogley et les sols peu évolués d'érosion ou sols sur cuirasse
ferrugineuse\,\cite{TIROGO_2017}.

\section{Point sur la situation générale de la dégradation des sols au Burkina Faso}

Pays sahélien, le Burkina Faso est soumis depuis plusieurs décennies à
une forte dégradation de ses ressources naturelles, limitant ainsi le
développement des productions
agro-sylvopastorales\,\cite{Thiombiano_2000}. Cela est en relation
avec le fait que le pays connaitrait en général des conditions
climatiques précaires, une croissance démographique relativement
élevée et une baisse continue de la fertilité des sols. Les sols étant
soumis à d’énormes contraintes climatiques (érosions hydrique et
éolienne, amplitudes thermiques, etc.), deviennent très sensibles à la
dégradation et ne supportent pas, de façon soutenue, les systèmes et
modes de production agricole actuellement
pratiqués\,\cite{PANA_2003}. A titre d’illustration, une étude de
\,\citeauthor{INERA_2003} montre que la dégradation affecte de nos jours plus de
24\,\% des terres arables au Burkina\,\cite{INERA_2003}, ce qui est
préjudiciable à l’économie nationale. Une autre étude plus récente estime
également qu’environ 11\,\% des terres du pays sont considérées comme
très dégradées et 34\,\%, comme moyennement dégradées\,\cite{SPCONEDD_2006}.
Les sols du Burkina Faso sont naturellement pauvres en
matières organiques notamment en éléments nutritifs essentiels dont
l’azote (N) et le phosphore (P)\,\cite{Traore_2008} et ont
une faible résistance à l’érosion\,\cite{Berger_1991}. Aussi, les
pays sahéliens en général et le Burkina en particulier sont sensibles,
vulnérables et en proie à une diminution accélérée des ressources
naturelles et à une aggravation de la pauvreté dans les zones
rurales\,\cite{Roose_2004}. Les aspects les plus ressentis par les
populations restent la réduction significative de la couverture
végétale avec une crise du
bois\footnote{https://www6.inra.fr/fabatropimed/FTM-Publications/FTM-Master-BTS}.
Au Burkina Faso, étant donné que moins de 10\,\% de la population a
accès à d’autres sources d’énergie que le bois de feu, près de 250\,000
hectares de forêts sont défrichés annuellement pour satisfaire les
besoins en bois de chauffe et 75\,000 hectares supplémentaires sont
convertis en nouveaux champs. Cette tendance est toujours à
l’augmentation. Dans le même temps, seuls 1\,000 hectares sont
reboisés\footnote{\url{http://www.jle.com/fr/revues/sec/e-docs/bois_de_feu_et_deboisement_au_sahel_mise_au_point_265802/article.phtml?tab=texte}}. A
titre illustratif, les Mossis, sur les plateaux du Burkina Faso,
brûlent les pailles du mil,faute de bois, et appauvrissent leurs sols
déjà épuisés. Les rendements s’en ressentent, et la solution à court
terme consistant à écourter les périodes de jachère ne pouvant ainsi
avoir que des conséquences catastrophiques à long
terme\footnote{\url{http://archives-fig-st-die.cndp.fr/actes/actes_2007/bret/article.htm}}.
A coté de la crise de bois, s’ajoutent la dégradation des sols avec la
perte de potentialités (d’où une chute des rendements agricoles) et
une diminution des ressources en eaux avec l’assèchement et
l’ensablement des cours d’eau\,\cite{ZOMBRE_2006}.

\section{Avantage des pratiques culturales d'associations et de rotations de légumineuses-céréales:}

Les associations ou rotations de cultures sont des stratégies qui
s’inscrivent bien dans le cadre de l’intensification écologique
puisqu’elles visent à améliorer la productivité par unité de surface
cultivée sans recours à des intrants supplémentaires. Ce sont des
systèmes bien adaptés à la petite agriculture familiale, car ils
peuvent permettre de mieux protéger les sols, d’en améliorer la
fertilité à travers notamment la fixation symbiotique de l’azote par
les légumineuses, de faciliter le contrôle des mauvaises herbes. Les
associations de culture sont aussi une stratégie de minimisation des
risques pour des exploitations soumises à des températures élevées et
des précipitations faibles et souvent fluctuantes. Optimiser le
fonctionnement de ces systèmes qui sont déjà largement utilisés par
les paysans est donc un enjeu fort de recherche développement.


\subsection{Avantages et intérêts généraux}

\paragraph{Associations culturales}


Les deux raisons les plus souvent avancées pour l’adoption des
associations d’espèces résident dans le gain de rendement global par
rapport à des cultures pures (mono spécifiques) et dans l’amélioration
significative et quasi systématique de la teneur en protéines de la
céréale, ceci quelle que soit sa proportion dans le mélange récolté
\,\cite{Jensen_1996}. Les associations légumineuse-céréale permettent
aussi une meilleure valorisation des ressources du milieu
comparativement aux cultures mono-spécifiques correspondantes dans les
systèmes à bas niveaux d’intrants azotés\,\cite{Bedoussac_2010}. Les
associations sont également un moyen de réduire, dans certaines
situations, la pression des adventices, maladies et ravageurs
\,\cite{Altieri_1999}, souvent considérés comme des facteurs
déterminants de la production agricole, faisant ainsi des associations
une alternative positive à la lutte
chimique\,\cite{Hauggaard_2001}. Les mélanges d’espèces présentent
aussi d’autres avantages comme : i) la réduction de l’érosion des sols
par une meilleure couverture et enracinement\,\cite{Zougmore_1998};
ii) l’amélioration de la résistance à la verse qui est un accident de
végétation atteignant principalement les céréales, provoqué par la
pluie, le vent ou une attaque de parasites et couchant les tiges au
sol\,\cite{Anil_1998}; iii) la réduction des risques de lixiviation
qui traduit la perte de nutriments (nitrates) végétaux hydrosolubles
du sol, qui sont dissous et entraînés par les eaux de pluie ou
d’irrigation\,\cite{CorreHellou_2005} ou encore, iv) la meilleure
stabilité inter-annuelle des rendements \,\cite{Lithourgidis_2006}. En
outre,\,\citeauthor{Chu_2004}\,\citeyear{Chu_2004} ont aussi montré
que le système de culture intercalaire est très prometteur pour le
développement de la production alimentaire durable dans les conditions
de ressources naturelles limitées, et en particulier dans les
situations de ressources limitées en eau\,\cite{Tsubo_2005}. Dans le
cas spécifique d’association du niébé avec le sorgho, la plante du
sorgho aide celle du niébé à se développer. De ce fait, les
associations culturales pourraient présenter un intérêt aussi bien en
agriculture biologique qu’en agriculture conventionnelle pour: i)
améliorer la qualité des composantes; ii) réduire les intrants
chimiques; iii) et améliorer les performances économiques et
environnementales des systèmes de
production\,\cite{Koulibi_FideleZONGO}.

%%  %insérer image

\begin{figure}%
  \begin{center}
   \includegraphics[width=16cm]{images/AssociationChampReel}
\end{center}
\caption{Représentation de l'association culturale en champ réel}
\end{figure}

\paragraph{Rotations culturales}

Les rotations de culture consistent à cultiver deux espèces avec un
intervalle de temps donné. Il s'agit de faire une suite de cultures
échelonnées au fil des années sur une même parcelle. En effet, le
maintien et l’amélioration de la fertilité des sols exigent
l’utilisation des engrais organiques et minéraux. Cependant les moyens
économiques des paysans ne permettent pas des interventions reposant
sur des investissements importants. Il devient alors nécessaire de
mettre en oeuvre des moyens naturels de gestion de la fertilité des
sols, comme la fixation biologique de l’azote atmosphérique par les
légumineuses dans un système de rotation des cultures
\footnote{\url{https://vtechworks.lib.vt.edu/bitstream/handle/10919/68409/4157_these_Traore_soil_fertility_2009.pdf?sequence=1}}. Les rotations de cultures constituent un élément
important de la gestion de la fertilité des sols et des bioagresseurs,
et donc représentent un atout pour l’augmentation des rendements.



\subsection{Biodisponibilité de l’azote dans le sol}

Par leur capacité à fixer l’azote atmosphérique grâce au processus de
la fixation symbiotique, les légumineuses améliorent le bilan de
l'azote dans les systèmes de
cultures\,\cite{Ndakidemi_2005,Fustec11}. Lors de la fixation d’azote
atmosphérique, les légumineuses contribuent à la teneur en azote (N)
du sol sous forme organique ou
minérale\,\footnote{\url{http://fertilisation-edu.fr/cycles-bio-geo-chimiques/le-cycle-de-l-azote-n.html}}
(nitrée) soit en tant que cultures pures en rotation ou, en
association \,\cite{Bado_2006,Chu_2004,Makoi_2009,Ndakidemi_2005}. En
culture associée par exemple, les légumineuses peuvent augmenter la
teneur en azote du sol grâce à la fixation et
l’excrétion\,\cite{Trenbath_1976,Fustec11}. De ce fait, les plantes
compagnes bénéficient de la fixation biologique par les légumineuses
et le transfert ultérieur de l’azote des légumineuses aux
non-légumineuses. Plusieurs travaux scientifiques ont signalé que le
transfert de l’azote fixé par les légumineuses symbiotiques à la
céréale se fait par l’intermédiaire de la décomposition et de la
minéralisation des racines des légumineuses et/ou des nodules
\,\cite{Burity_1989} et par les exsudats racinaires libérés des
légumineuses\,\cite{Ndakidemi_2005,Makoi_2009}. Ce processus est
appelé rhizodéposition\cite{Fustec11} et constitue une source
potentielle d’accumulation de l’azote dans le
sol\,\cite{Koulibi_FideleZONGO}. Le niébé est une espèce généralement
efficace dans la fixation de l’azote grâce aux rhizobiums associés à
ses
racines\footnote{\url{https://vtechworks.lib.vt.edu/bitstream/handle/10919/68409/4157_these_Traore_soil_fertility_2009.pdf?sequence=1}}.

\newpage

 \bibliographystyle{apacite}
  \bibliography{revue_bibliographique}

\end{document}

\documentclass[a4paper,11pt]{article}

\usepackage[utf8]{inputenc} \usepackage{mathpazo} \usepackage{gensymb}
\usepackage{graphicx} \usepackage{url} \usepackage[french]{babel}
\usepackage{subfig} \usepackage{apacite} \usepackage{array}
\usepackage{multirow} \usepackage{dcolumn}
\newcolumntype{d}[1]{D{,}{,}{#1}} \usepackage[T1]{fontenc}
\usepackage{amsmath}

\begin{document}

\title{Fastqc \& Multiqc review} \author{Marilyne Aza-Gnandji}
\date{\today}

\maketitle \tableofcontents

\section{Généralités}

FastQC est un logiciel qui a été développé par Simon
Andrews\footnote{{https://github.com/s-andrews/FastQC}} et permet de
faire une analyse de la qualité de données de séquençage NGS (ADN/
ARN). En effet, les séquenceurs modernes à haut débit peuvent générer
des dizaines de millions de séquences en une seule fois.  Avant
l'analyse proprement dite de ces séquences, afin de tirer des
conclusions biologiques, il est essentiel d'effectuer des contrôles de
qualité simples. Généralement, cette étape est effectuée par le
prestataire. Néanmoins grâce à FastQC, il est possible de déterminer
par soi-même, entre autres :
\begin{itemize}
\item[\textbullet] le nombre de lectures (reads)
\item[\textbullet] la qualité des lectures
\item[\textbullet] la longueur des lectures
\item[\textbullet] une éventuelle contamination
\item[\textbullet] la présence suspecte d'adaptateurs
\item[\textbullet]
  etc\footnote{{http://genome.jouy.inra.fr/tutorials/QualityControl/quality_control.html}}.
\end{itemize}

Il s'agit de s'assurer que dans les données brutes obtenues, il n'y a
pas de problèmes ni de biais qui pourraient influer sur les
connaissances que l'on cherche à extraire de leurs exploitations. La
plupart des séquenceurs génèrent un rapport de contrôle qualité dans
le cadre de leur pipeline d’analyses, mais cela ne concerne
généralement que l’identification des problèmes générés par le
séquenceur lui-même.  FastQC a pour objectif de fournir un rapport de
contrôle qualité capable de détecter les problèmes dans le séquenceur
ou dans la bibliothèque de départ. FastQC peut être exécuté dans l'un
des deux modes. Il peut soit fonctionner comme une application
interactive autonome pour l’analyse immédiate de petits nombres de
fichiers FastQ, ou il peut être exécuté dans un mode non interactif où
il conviendrait de l’intégrer dans un pipeline d’analyses plus grand
pour le traitement systématique d’un grand nombre de
fichiers\footnote{\url{https://dnacore.missouri.edu/PDF/FastQC_Manual.pdf}}.
FastQC lit un ensemble de fichiers de séquences et produit à partir de
chacun d'eux un rapport de contrôle de la qualité composé d'un certain
nombre de modules différents. Chaque module permettra d'identifier un
type de problème potentiel sur les données.Le logiciel prend en entrée
des fichiers au format sam, bam et fastq. Il faut dire que les données
sont généralement fournies au format FASTQ et si ce n'est pas le cas,
il existe des outils de conversion pour obtenir des fichiers
FASTQ\footnote{\url{http://genome.jouy.inra.fr/tutorials/QualityControl/quality_control.html}}. Le
logiciel peut aussi lire directement les fichiers fastq.gz.  Dans le
cadre de nos analyses avec 80 échantillons, nous avons choisi la
deuxième option d'utilisation du logiciel i.e son intégration dans un
pipeline d'analyses.

\section{Mode d'utilisation}

\subsection{Principales fonctions}

\begin{itemize}
\item Importation de données à partir de fichiers BAM, SAM ou FastQ
  (toute variante);
\item Fournir un aperçu rapide pour vous dire dans quels domaines il
  peut y avoir des problèmes;
\item Des graphiques et des tableaux récapitulatifs pour évaluer
  rapidement vos données;
\item Exportation des résultats dans un rapport permanent basé sur
  HTML;
\item Opération hors ligne pour permettre la génération automatisée de
  rapports sans exécuter l'application
  interactive\footnote{\url{http://www.bioinformatics.babraham.ac.uk/projects/fastqc/}}.
\end{itemize}

\subsection{Exemples de résultats et interprétations}

Pour un usage généraliste, FastQC est particulièrement adapté. Il
permet une visualisation graphique des différentes métriques d'intérêt
avec un code
couleur\footnote{\url{http://genome.jouy.inra.fr/tutorials/QualityControl/quality_control.html}}. Les
figures 1 à 7 sont une illustration de quelques informations que l'on
peut obtenir en utilisant FastQC:

La Fig.~\ref{fig-Fastqc_Plots1} p.~\pageref{fig-Fastqc_Plots1} montre
une représentation dans le mauvais cas et dans le bon cas de la
distribution par base du score de qualité. Dans les lectures (reads)
générés par Illumina, la qualité diminue généralement vers l'extrémité
3'.

\begin{figure}
  \begin{center}
    \includegraphics[width=16cm]{Images/Fastqc_Plots1}
  \end{center}
  \caption{Qualité de séquence par base}
  \label{fig-Fastqc_Plots1}
\end{figure}

La représentation de la Fig.~\ref{fig-Fastqc_Plots2}
p.~\pageref{fig-Fastqc_Plots2} nous permet de détecter un
sous-ensemble de séquences de basse qualité et de définir un seuil de
qualité.

\begin{figure}
  \begin{center}
    \includegraphics[width=16cm]{Images/Fastqc_Plots2}
  \end{center}
  \caption{Scores de qualité par séquence}
  \label{fig-Fastqc_Plots2}
\end{figure}

La représentation de la Fig.~\ref{fig-Fastqc_Plots3}
p.~\pageref{fig-Fastqc_Plots3} nous permet de détecter la présence
d'adaptateurs ou de mettre en évidence une fragmentation biaisée.

\begin{figure}
  \begin{center}
    \includegraphics[width=16cm]{Images/Fastqc_Plots3}
  \end{center}
  \caption{Contenu en séquences par base}
  \label{fig-Fastqc_Plots3}
\end{figure}

Dans la Fig.~\ref{fig-Fastqc_Plots4} p.~\pageref{fig-Fastqc_Plots4},
nous notons des pics en contenu GC, ce qui traduit une éventuelle
contamination.

\begin{figure}
  \begin{center}
    \includegraphics[width=16cm]{Images/Fastqc_Plots4}
  \end{center}
  \caption{Contenu en GC par séquence}
  \label{fig-Fastqc_Plots4}
\end{figure}

Les séquences sur-représentées comme illustrées dans la
Fig.~\ref{fig-Fastqc_Plots5} p.~\pageref{fig-Fastqc_Plots5} sont une
indication d'une possible source de contamination.

\begin{figure}
  \begin{center}
    \includegraphics[width=16cm]{Images/Fastqc_Plots5}
  \end{center}
  \caption{Les séquences sur-représentées}
  \label{fig-Fastqc_Plots5}
\end{figure}

La Fig.~\ref{fig-Fastqc_Plots6} p.~\pageref{fig-Fastqc_Plots6} nous
indique que nous sommes face à un mauvais cas de figure i.e la
représentation de gauche qu'un niveau élevé de duplication est
susceptible d'indiquer un biais d'enrichissement (par exemple, une
suramplification de la PCR).

\begin{figure}
  \begin{center}
    \includegraphics[width=16cm]{Images/Fastqc_Plots6}
  \end{center}
  \caption{Niveaux de duplication par séquence}
  \label{fig-Fastqc_Plots6}
\end{figure}

Sur la Fig.~\ref{fig-Fastqc_Plots7} p.~\pageref{fig-Fastqc_Plots7},
est illustrée à gauche les problèmes potentiels des cellules
d'écoulement (flowcell) d'Illumina. La flowcell est la partie d'un
cytomètre en flux dans laquelle un flux de cellules est aligné pour
traverser une fois à la fois un faisceau lumineux.

\begin{figure}
  \begin{center}
    \includegraphics[width=16cm]{Images/Fastqc_Plots7}
  \end{center}
  \caption{Qualité des séquences par cadrant}
  \label{fig-Fastqc_Plots7}
\end{figure}


\newpage

MultiQC a été développé par Phil
EWELS\footnote{\url{https://github.com/ewels/MultiQC/tree/master/docs}}. C'est
un outil permettant d’agréger les résultats de la bioinformatique sur
de nombreux échantillons dans un seul rapport. Dans notre cas, nous
utilisons MultiQC pour visualiser les métriques issues de FastQC de
tous les échantillons en un seul rapport (fichier .html).  A partir de
nos données d'étude, nous présentons quelques informations que MultiQC
fournit:

\begin{figure}
  \begin{center}
    \includegraphics[width=16cm]{Images/MultiQC_Plots1}
  \end{center}
  \caption{Histogramme de qualité des séquences de
    \og{}paired-end-reads\fg{} de 79 échantillons}
  \label{fig-MultiQC_Plots1}
\end{figure}

La Fig.~\ref{fig-MultiQC_Plots1} p.~\pageref{fig-MultiQC_Plots1} nous
montre la valeur moyenne de qualité des séquences à travers chaque
position de base des lectures (reads) et nous indique que le score de
qualité de toutes nos séquences de reads est supérieure à 30. Cela
signifie qu'il n'y a pas de problèmes potentiels avec les bases
constituant nos séquences. En effet, le score de qualité est relié de
façon logarithmique à la probabilité d'erreur. Cette probabilité
d'erreur est calculée lors de l'identification d'une base et
correspond à la probabilité qu'il s'agisse d'une mauvaise
identification. Concrètement, un score de 20 signifie qu'il existe une
chance sur 100 pour que la base soit fausse. Un score de 30 à une
chance sur 1000.

MultiQC présente au travers de la représentation de la
Fig.~\ref{fig-MultiQC_Plots2} p.~\pageref{fig-MultiQC_Plots2}, le
nombre de lectures (reads) avec la moyenne respective des scores de
qualité.

\begin{figure}
  \begin{center}
    \includegraphics[width=16cm]{Images/MultiQC_Plots2}
  \end{center}
  \caption{Score de qualité par séquence}
  \label{fig-MultiQC_Plots2}
\end{figure}

MultiQC présente au travers de la représentation de la
Fig.~\ref{fig-MultiQC_Plots3} p.~\pageref{fig-MultiQC_Plots3}, la
proportion de chaque position de base pour laquelle chacune des quatre
bases d'ADN normales a été appelée.

\begin{figure}
  \begin{center}
    \includegraphics[width=16cm]{Images/MultiQC_Plots3}
  \end{center}
  \caption{Contenu en bases par séquence}
  \label{fig-MultiQC_Plots3}
\end{figure}

MultiQC présente au travers de la représentation de la
Fig.~\ref{fig-MultiQC_Plots4} p.~\pageref{fig-MultiQC_Plots4}, le
contenu moyen en GC des lectures.

\begin{figure}
  \begin{center}
    \includegraphics[width=16cm]{Images/MultiQC_Plots4}
  \end{center}
  \caption{Représentation du contenu en GC par lecture}
  \label{fig-MultiQC_Plots4}
\end{figure}

Au travers de la représentation de la Fig.~\ref{fig-MultiQC_Plots5}
p.~\pageref{fig-MultiQC_Plots5}, MultiQC nous renseigne sur le contenu
en \og{}N\fg{} par séquence de reads. L'indication de N signifie qu'on
ne sait exactement laquelle des bases (A,T,C,G) est concernée.

\begin{figure}
  \begin{center}
    \includegraphics[width=16cm]{Images/MultiQC_Plots5}
  \end{center}
  \caption{Contenu en \og{}N\fg{} par séquence}
  \label{fig-MultiQC_Plots5}
\end{figure}

La Fig.~\ref{fig-MultiQC_Plots6} p.~\pageref{fig-MultiQC_Plots6},
renseigne sur le niveau relatif de duplication trouvé pour chaque
séquence.

\begin{figure}
  \begin{center}
    \includegraphics[width=16cm]{Images/MultiQC_Plots6}
  \end{center}
  \caption{Niveaux de duplication par séquence}
  \label{fig-MultiQC_Plots6}
\end{figure}

La Fig.~\ref{fig-MultiQC_Plots7} p.~\pageref{fig-MultiQC_Plots7},
indique le contenu en adaptateurs.

\begin{figure}
  \begin{center}
    \includegraphics[width=16cm]{Images/MultiQC_Plots7}
  \end{center}
  \caption{Contenu en adaptateurs}
  \label{fig-MultiQC_Plots7}
\end{figure}

\end{document}

\documentclass[a4paper,11pt]{article}

\usepackage[utf8]{inputenc}
\usepackage{mathpazo}
\usepackage{gensymb}
\usepackage{graphicx}
\usepackage{url}
\usepackage[frenchb]{babel}
\usepackage{subfig}
\usepackage{apacite}
\usepackage{array}
\usepackage{multirow}
\usepackage{dcolumn}
\newcolumntype{d}[1]{D{,}{,}{#1}}
\usepackage[T1]{fontenc}
\usepackage{amsmath}

\begin{document}

\title{Fastqc \& Multiqc review}
\author{Marilyne Aza-Gnandji}
\date{\today}

\maketitle
\tableofcontents

\section{Généralités}

FastQC est un logiciel qui a été développé par Simon
Andrews\footnote{{https://github.com/s-andrews/FastQC}} et permet de
faire une analyse de la qualité de données de séquençage NGS (ADN/
ARN). En effet, les séquenceurs modernes à haut débit peuvent générer
des dizaines de millions de séquences en une seule fois.  Avant
l'analyse proprement dite de ces séquences, afin de tirer des
conclusions biologiques, il est essentiel d'effectuer des contrôles de
qualité simples.  Il s'agit de s'assurer que dans les données brutes
obtenues, il n'y a pas de problèmes ni de biais qui pourraient influer
sur les connaissances que l'on cherche à extraire de leurs
exploitations. La plupart des séquenceurs génèrent un rapport de
contrôle qualité dans le cadre de leur pipeline d’analyses, mais cela
ne concerne généralement que l’identification des problèmes générés
par le séquenceur lui-même. FastQC a pour objectif de fournir un
rapport de contrôle qualité capable de détecter les problèmes dans le
séquenceur ou dans la bibliothèque de départ. FastQC peut être exécuté
dans l'un des deux modes. Il peut soit fonctionner comme une
application interactive autonome pour l’analyse immédiate de petits
nombres de fichiers FastQ, ou il peut être exécuté dans un mode non
interactif où il conviendrait de l’intégrer dans un pipeline
d’analyses plus grand pour le traitement systématique d’un grand
nombre de
fichiers\footnote{\url{https://dnacore.missouri.edu/PDF/FastQC_Manual.pdf}}.
FastQC lit un ensemble de fichiers de séquences et produit à partir de
chacun d'eux un rapport de contrôle de la qualité composé d'un certain
nombre de modules différents. Chaque module permettra d'identifier un
type de problème potentiel sur les données. Le logiciel prend en
entrée des fichiers au format sam, bam et fastq. Il faut dire que les
données sont généralement fournies au format FASTQ et si ce n'est pas
le cas, il existe des outils de conversion pour obtenir des fichiers
FASTQ\footnote{{http://genome.jouy.inra.fr/tutorials/QualityControl/quality_control.html}}. Le
logiciel peut aussi lire directement les fichiers .fastq.gz.  Dans le
cadre de nos analyses avec 80 échantillons, nous avons choisi la
deuxième option d'utilisation du logiciel i.e son intégration dans un
pipeline d'analyses.

\section{Mode d'utilisation}

\subsection{Principales fonctions}

\begin{itemize}
  \item Importation de données à partir de fichiers BAM, SAM ou FastQ
    (toute variante);
   \item Fournir un aperçu rapide pour vous dire dans quels domaines
     il peut y avoir des problèmes;
    \item Des graphiques et des tableaux récapitulatifs pour évaluer
      rapidement vos données;
     \item Exportation des résultats dans un rapport permanent basé
       sur HTML;
      \item Opération hors ligne pour permettre la génération
        automatisée de rapports sans exécuter l'application
        interactive.
\end{itemize}

\end{document}
